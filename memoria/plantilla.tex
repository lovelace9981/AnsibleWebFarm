\input{preambulo.tex}

%----------------------------------------------------------------------------------------
%	TÍTULO Y DATOS DEL ALUMNO
%----------------------------------------------------------------------------------------

\title{
\normalfont \normalsize
\textsc{\textbf{Servidores Web y Alta Disponibilidad (2021-2022)} \\ Grado en Ingeniería Informática \\ Universidad de Granada} \\ [25pt] % Your university, school and/or department name(s)
\horrule{0.5pt} \\[0.4cm] % Thin top horizontal rule
\huge Memoria Trabajo Final \\ % The assignment title
\horrule{2pt} \\[0.5cm] % Thick bottom horizontal rule
}

%https://es.overleaf.com/learn/latex/Inserting_Images
%Ruta relativa de   imagenes
\graphicspath{ {img/} }

\author{Pedro Antonio Mayorgas Parejo y Adrián Acosa Sánchez} % Nombre y apellidos

\date{\normalsize\today} % Incluye la fecha actual

%----------------------------------------------------------------------------------------
% DOCUMENTO
%----------------------------------------------------------------------------------------

\begin{document}

\maketitle % Muestra el Título

\newpage %inserta un salto de página

\tableofcontents % para generar el índice de contenidos

\newpage

%----------------------------------------------------------------------------------------
%	Cuestión 1
%----------------------------------------------------------------------------------------

\section{Abstract}

% \vspace{5mm}


% \begin{lstlisting}[style=mybash]
%     # Para una base de datos concreta
%     mysqldump --user=tiendabd --password=password --databases tiendabd --add-drop-database --add-drop-table [--replace] --host=127.0.0.1 --result-file=dump.sql
% \end{lstlisting}



%\begin{figure}[H]
%	\centering
%	\includegraphics[scale=0.30]{cuestion_1_1}
%	\caption{Se puede ver que al no haber un fallo grave, el sistema lo nota como que sigue funcionando pero en un estado degradado.}
%\end{figure}

%\newpage

%Se pueden hacer include en latex
%\input{plantilla_include.tex}


%-------Bibliografia-----------------------------

%\newpage
\section{Bibliografía}

% Ejemplo
\footnote{Instalación de Vagrant}
\textcolor{blue}{\url{https://www.vagrantup.com/downloads}}

\footnote{Vagrant Libvirt provider}
\textcolor{blue}{\url{https://github.com/vagrant-libvirt/vagrant-libvirt}}

\footnote{Instalación de Ansible}
\textcolor{blue}{\url{https://docs.ansible.com/ansible/latest/installation\_guide/intro\_installation.html\#installing-ansible-on-debian}}

\footnote{Referencia y instalación de community.Crypto para crear certificados CA y para HTTPs}
\textcolor{blue}{\url{https://docs.ansible.com/ansible/latest/collections/community/crypto/index.html}}

\footnote{Referencia y instalación de community.Mysql para crear gestionar bases de datos con Ansible}
\textcolor{blue}{\url{https://docs.ansible.com/ansible/latest/collections/community/mysql/index.html}}

\footnote{Enlace al repositorio de las memorias necesarias para comprender el background completo}
\textcolor{blue}{\url{https://github.com/lovelace9981/MemoriasInvestigacionSWAP}}

\footnote{Enlace al repositorio del proyecto completo (se pondrá a público en cuanto se sepa la nota y termine la convocatoria)}
\textcolor{blue}{\url{https://github.com/lovelace9981/AnsibleWebFarm}}


\end{document}
